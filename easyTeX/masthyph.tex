% masthyph.tex (Aug. 1990)
% Copyright Michael J. Ferguson, INRS-Telecommunications
% All rights reserved.

% Master Multilingual Pattern Inputs

% This file defines the changes in the \lccodes needed for Multilingual TeX
% and then inputs the patterns for English and French. 
% it uses the same pattern set as that used in Multilingual TeX ...
% you must input the "extdef.tex" conversions before the patterns are processed

% Special lccodes and hyphenation for Multilingual Version

\gdef\accenthyphcodes{
    \def\oe{^^[} % \oe 
    \def\i{^^P}
    \def\'##1{\csname @ac@##1\endcsname}
    \def\`##1{\csname @gr@##1\endcsname}
    \def\v##1{\csname @v@##1\endcsname}
    \let\^^_=\v
    \def\u##1{\csname @u@##1\endcsname}
    \let\^^S=\u
    \def\=##1{\csname @eq@##1\endcsname}
    \def\^##1{\csname @hat@##1\endcsname}
    \let\^^D=\^
    \def\.##1{\csname @dot@##1\endcsname}
    \def\H##1{\csname @H@##1\endcsname}
    \def\~##1{\csname @til@##1\endcsname}
    \def\"##1{\csname @um@##1\endcsname}
    \let\c@@=\c
    \def\c##1{\csname c@##1\endcsname}}

\gdef\spechyphcodes{}

% Pattern Input / French and English %%%%%%%%%%%%%%%%%%%%%%%%%%%%%%%%%%%%%%%%%%%%%%%

% French hyphenation patterns
\begingroup
\language=1
\input frhyph \relax
\endgroup
% definitions for fast french hyphenation
\def\fhyph{\language=1 \lccode`\'=`\'\frenchspacing}

% English hyphenation exceptions
\begingroup
\language=0
\input enhyphex \relax   % exceptions
\input enhyph \relax     % pattern
\endgroup
% definitions for fast english hyphenation
\def\ehyph{\language=0 \lccode`\'=0 \nonfrenchspacing}
