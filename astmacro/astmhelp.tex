%%%%%%%%%%%%%%%%%%%%%%%%%%%%%%%%%%%%%%%%%%%%%%%%%%%%%%%%%%%%%%%%%%%%%%%%%%%%%%%%%%%%
% doc.macro.file{
%    filename          = "astmhelp.tex",
%    date              = "04/Oct/1998",
%    last revision     = "26/Feb/1999",
%    filetype          = "TeX",
%    author            = "Davide MADRISAN",
%    codetable         = "ISO/ASCII",
%    checksumtype      = "line count",
%    checksum          = "155",
%    abstract          = "This file is a tutorial for the macros package 
%                         'astmacro.tex', version 1.8.03m.
%                         Read it and look at the DVI".
% }
%%%%%%%%%%%%%%%%%%%%%%%%%%%%%%%%%%%%%%%%%%%%%%%%%%%%%%%%%%%%%%%%%%%%%%%%%%%%%%%%%%%%

\input astmacro

% (1) %%% FIRST PAGE MACROS AND ENVIRONMENTS %%%%%%%%%%%%%%%%%%%%%%%%%%%%%%%%%%%%%%%
% Example:
  \author{THE AUTHOR}
  \headlinetitle{THIS IS THE TITLE THE HEADLINE}

  \begin{doceye}
      Acc.~Sc.~Torino - Memorie Sc.~Fis.\\
      25 ~{\rm(}1998{\/\rm)}, 1-25, 3 ff, 2 tabb.
  \end{doceye}

  \subject{Subject of the paper}
  \title{Main title at the top of the first page}

  \begin{authornote}
      Memoria del Socio nazionale residente THE AUTHOR*\\
      presentata nell'adunanza del 1 Gennaio 1998
  \end{authornote}
 
  \begin{firstpagenote}
      1991 Mathematics Subject Classification. Primary 33C25.\\
      *Dipartimento di Matematica dell'Universit�, Torino, Italy.\\
      This work was supported by the Consiglio Nazionale delle Ricerche of Italy.
  \end{firstpagenote}
%%%%%%%%%%%%%%%%%%%%%%%%%%%%%%%%%%%%%%%%%%%%%%%%%%%%%%%%%%%%%%%%%%%%%%%%%%%%%%%%%%%%
%
% (2) %%% ABSTRACT and END OF ABSTRACT %%%%%%%%%%%%%%%%%%%%%%%%%%%%%%%%%%%%%%%%%%%%%
% Example:
  \begin{abstract}
      Abstract\\
      This paragraph is automatically formatted by the environment 
      {\tt\char92begin\lb abstract\rb$\dots$\char92end\lb abstract\rb}.
  \end{abstract}
 
  \medskip
  \begin{abstract}
      Riassunto\\
      The same environment (note the use of {\tt\char92\char92}\/).
  \end{abstract}
%%%%%%%%%%%%%%%%%%%%%%%%%%%%%%%%%%%%%%%%%%%%%%%%%%%%%%%%%%%%%%%%%%%%%%%%%%%%%%%%%%%%
%
% (3) %%% SECTION %%%%%%%%%%%%%%%%%%%%%%%%%%%%%%%%%%%%%%%%%%%%%%%%%%%%%%%%%%%%%%%%%%
% Examples: 
  \section{1. Introduction, using the macro {\tt\char92section}}

  Next ``Lemma'' and ``Proof'' are an example of the two environments:
  \itemitem{$-$} 
      {\tt\char92begin\lb proclaim\rb$\dots$\char92end\lb proclaim\rb},
  \itemitem{$-$} 
      {\tt\char92begin\lb proof\rb$\dots$\char92end\lb proof\rb}.
%%%%%%%%%%%%%%%%%%%%%%%%%%%%%%%%%%%%%%%%%%%%%%%%%%%%%%%%%%%%%%%%%%%%%%%%%%%%%%%%%%%%
%
% (4) %%% PROCLAIM, PROOF, MATHNO and VEOP, BEOP %%%%%%%%%%%%%%%%%%%%%%%%%%%%%%%%%%%
% Example: 
  \begin{proclaim}
      Lemma 1.1\\
      Let $v, w \in V$ be such that $w \in \gamma(v)$ holds almost everywhere in
      ${\Omega}$, where $\gamma:{{\rm I}\!{\rm R}} \to 2^{{{\rm I}\!{\rm R}}}$
      is a maximal monotone graph. Then
      $$
      \int_{\Omega} \nabla v \cdot \nabla w \geq 0.
      \mathno{1.1}
      $$

  \begin{proof}
      Proof\\
      Let $\mu > 0$ be arbitrary and note that 
      $w + \mu v \in (\mu {\cal I} + \gamma)(v)$.
      Then $v = (\mu {\cal I} + \gamma)^{-1} (w +\mu v)$.
      Since $(\mu {\cal I} + \gamma)^{-1}$ is Lipschitz continuous $\dots$
      $$
      \int_{\Omega} \nabla v \cdot \nabla (w + \mu v) =
      \int_{\Omega} \vert \nabla(w + \mu v)\vert^2
      \big((\mu {\cal I} + \gamma)^{-1}\big)^\prime (w + \mu v)  \geq 0. 
      \mathno{1.2}
      $$
      Letting $ \mu\searrow 0 $ in $\dots$
  \end{proof}
  \end{proclaim}
%
% \veop % void square point
% \beop % black square point
%
% (5) %%% SMALLSECTION %%%%%%%%%%%%%%%%%%%%%%%%%%%%%%%%%%%%%%%%%%%%%%%%%%%%%%%%%%%%%
  \begin{smallsection}
      Now the environment 
      {\tt\char92begin\lb smallsection\rb$\dots$\char92end\lb smallsection\rb}.\par
      This is quite useful for long quotations, and this paragraph is an 
      example of its use. 
      You can also use a {\tt\char92veop} (\veop)
      or a {\tt\char92beop} (\beop) to end your paragraphs.
  \end{smallsection}
%%%%%%%%%%%%%%%%%%%%%%%%%%%%%%%%%%%%%%%%%%%%%%%%%%%%%%%%%%%%%%%%%%%%%%%%%%%%%%%%%%%%
%
% (6) %%% FIGURE MACROS %%%%%%%%%%%%%%%%%%%%%%%%%%%%%%%%%%%%%%%%%%%%%%%%%%%%%%%%%%%%
% Example:
  \midspace{7cm}{Figure 1}
      {The macros {\tt\char92midspace\lb\rb\lb\rb\lb\rb} and 
      {\tt\char92topspace\lb\rb\lb\rb\lb\rb}}
% \topspace{7cm}{Figure 1}{The function $F(x)$}
%
% You can use the macros '\topfigure' and '\midfigure' to insert images in your
% documents (for instance (e)ps files generated with MapleV).
% For more information see the in the macros file astmacro, subsection 6.2.
% Example:
% \legendvoffset = 12pt plus 4pt minus 4pt
% \epstopfigure{figure.ps}{8cm}{4.5cm}{2.2cm}{1.1}{Legend of the image}
%%%%%%%%%%%%%%%%%%%%%%%%%%%%%%%%%%%%%%%%%%%%%%%%%%%%%%%%%%%%%%%%%%%%%%%%%%%%%%%%%%%%
%
% (7) %%% FOOTNOTE %%%%%%%%%%%%%%%%%%%%%%%%%%%%%%%%%%%%%%%%%%%%%%%%%%%%%%%%%%%%%%%%%
% Example:
  Now an example of a footnote at the bottom of the page
  \begin{footnote}
      I am a footnote.
  \end{footnote}
  using the environment 
  {\tt\char92begin\lb footnote\rb$\dots$\char92end\lb footnote\rb}.
%%%%%%%%%%%%%%%%%%%%%%%%%%%%%%%%%%%%%%%%%%%%%%%%%%%%%%%%%%%%%%%%%%%%%%%%%%%%%%%%%%%%
%
% (8) %%% REFERENCES (BIBLIOREF and BIBITEM) %%%%%%%%%%%%%%%%%%%%%%%%%%%%%%%%%%%%%%%
% Example:
  \section{2. The use of {\tt\char92biblioref\lb\rb\lb\rb}}

  We consider the following two-term uniform asymptotic representation, due to 
  \biblioref{Olver}{10, 11, 1 (9.3.35)} $\dots$

  \section{References, or the macro {\tt\char92bibitem\lb\rb\lb\rb}}

  \bibitem{1}{J.P.~Aubin {\rm ~and~} I.~Ekeland,}
      {Applied nonlinear analysis,} 
      {Pure Appl. Math. {\bf 64}, Wiley, New York 1984.}

  \bibitem{2}{V. Barbu,}
      {Nonlinear semigroups and differential equations in Banach spaces,}
      {Noordhoff International Publishing, Leyden 1976.}
%%%%%%%%%%%%%%%%%%%%%%%%%%%%%%%%%%%%%%%%%%%%%%%%%%%%%%%%%%%%%%%%%%%%%%%%%%%%%%%%%%%%

\bye