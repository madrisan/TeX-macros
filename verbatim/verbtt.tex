%%%%%%%%%%%%%%%%%%%%%%%%%%%%%%%%%%%%%%%%%%%%%%%%%%%%%%%%%%%%%%%%%%%%%%%%%%%%%%%%%%%%
% @TeX-file{
%    author            = "Davide MADRISAN",
%    filename          = "verbtt.tex",
%    version           = "3.4b",
%    date              = "16/Mag/1999",
%    copyright         = "This file copyright by Davide Madrisan",
%    codetable         = "ISO/ASCII",
%    checksumtype      = "line count",
%    checksum          = "237",
%    dependencies      = "plain.tex",
%    abstract          = "This package implements a verbatim environment. It is
%                         intended to offer a good interface to programmers that
%                         need to insert verbatim text or files into TeX documents".
% }
%%%%%%%%%%%%%%%%%%%%%%%%%%%%%%%%%%%%%%%%%%%%%%%%%%%%%%%%%%%%%%%%%%%%%%%%%%%%%%%%%%%%

% RESTRICTIONS: the verbatim macros are provided free of charge without warranty of
% any kind. They may be distributed freely provided that the recipients also acquire
% the right to distribute them freely. The notices to this effect must be preserved
% when the files are distributed. If you want to change something, change the 
% filename from 'verbtt.tex' to something else and indicate any changes in the file
% itself, or else remove my name from the file completely.

\def\shouldbefmt{plain}        % controllo precauzionale del formato di compilazione
\newlinechar = `\^^J
\def\lookatformat{                   % this package requires plain TeX to run safety
    \ifx\shouldbefmt\fmtname%
    \else\immediate\write16{^^J%
        > !OOPS... warning! - I'm not running under Plain TeX!^^J}\fi}\lookatformat

% (1) %%% verbatim environment for \tt (Computer Modern) fonts %%%%%%%%%%%%%%%%%%%%%
% Esempio d'uso della macro verbatim:
%
%    Qualsiasi testo che si richiede venga formattato da \TeX.
%    \verbatim
%    matrix::~matrix() { // distruttore della classe matrix
%       if(--p->n == 0) { // se il contatore di riferimenti diventa 0
%          for(int r = 0; r < rows(); r++)
%             delete p->m[r];
%          delete p->m;
%          delete p;
%       }
%    }
%    @endverbatim
%    \noindent Ritorno al trattamento di default del testo.
% 
% L'ambiente verbatim permette l'inserimento in documenti TeX di testo comprendente
% anche i caratteri speciali TeX, che si desidera vengano trattati da TeX in modo
% letterale, cioe' non interpretati. I caratteri supportati dall'ambiente verbatim
% sono riportati nella seguente tabella:
%
%    carattere    \catcode    funzione
%   ------------------------------------------------------------------------
%     `\\            0         backslash is TeX escape character
%     `\{            1         left brace is begin-group character
%     `\}            2         right brace is end-group character
%     `\$            3         dollar sign is math shift
%     `\&            4         ampersand is alignment tab
%     `\^^M          5         ascii return is end-line
%     `\#            6         hash mark is macro parameter character
%     `\^   `\^^K    7         circumflex and uparrow are for superscripts
%     `\_   `\^^A    8         underline and downarrow are for subscripts
%     `\^^I `\      10         ascii tab and ascii space are blank spaces
%     `\~           13         tilde
%     `\%           14         percent sign is comment character
%
% Per impedire a TeX di considerare questi caratteri come speciali ho modificato i
% loro '\catcode', impostandoli tutti al valore '12'. Per far questo in blocco ho
% utilizzato la macro plain-TeX '\dospecials':
%
%    \def\dospecials{%
%       \do\ \do\\\do\{\do\}\do\$\do\&\do\#\do\^\do\^^K\do\_\do\^^A\do\%\do\~}
%
% In essa sono elencati i carattere speciali, preceduti dal comando '\do', che ho
% definito (localmente) in '\uncatcodespecials' in modo da effettuare con tale macro
% la conversione dei '\catcode' come indicato sopra:
%
%   \chardef\other = 12 % '\other' e' ora sinonimo di '12'
%   \def\uncatcodespecials{\def\do##1{\catcode`##1 = \other}\dospecials}
%
% Per quando riguarda i caratteri restanti (tabulazione e return) ho usato la linee:
%
%   \chardef\active = 13 % (definito in 'plain.tex')
%   \def\makeactive#1{\catcode`#1 = \active\ignorespaces}
%
%   \bgroup      % il gruppo delimita le linee di testo in cui sono attivi ^^M e ^^I
%       \makeactive\^^M \makeactive\^^I %
%       \gdef\setupverbatim{%
%           \makeactive\^^M \let^^M = \newline%
%           \makeactive\^^I \let^^I = \tabtospaces        % conversione tab in spazi
%           \catcode`\@ = 0 \uncatcodespecials%
%           \parskip = 0pt \baselineskip = \verbatimbaselineskip        % interlinea
%           \obeyspaces\frenchspacing}                         % spaziatura francese
%   \egroup
%
% Esse definiscono globalmente ('\gdef') la macro '\setupverbatim', che si occupa di
% rendere letterale il trattamento dei caratteri di spaziatura (cioe' di inserire
% nel file DVI gli spazi nelle quantita' e modalita' indicate nel testo verbatim),
% di cambiare il '\catcode' dei due caratteri al valore '\active' e di ridefinire i
% caratteri '^^M' (return) e '^^I' (tab) rendendoli rispettivamente sinonimi di
% '\newline' e '\tabtospaces'.
% Le definizioni seguenti sono naturalmente personalizzabili:
%
%   \def\newline{\endgraf\noindent}
%   {\obeyspaces\gdef\tabtospaces{\ \ \ }} % un tab equivale a 3 spazi
% 
% ad esempio:
%
%   \def\newline{\endgraf\indent}
%
% indica a TeX di indentare le linee verbatim, mentre:
%
%   \newcount\linenum\linenum = 0 % contatore per numerazione linee listato
%   \def\newline{%
%       \advance\linenum by 1
%       \endgraf\noindent\leavevmode\llap{\sevenrm\the\linenum.\ \ }}
%
% numera (in modo progressivo, iniziando da 1) le linee del testo verbatim (!).
% Le restanti linee del blocco '\bgroup' .. '\egroup' prededente:
%
%    \catcode`\@ = 0
%    \parskip = 0pt \baselineskip = \verbatimbaselineskip
%    \obeyspaces\frenchspacing
% 
% settano le modalita' di spaziatura e l'interlinea del testo verbatim.
% A questo proposito, '\verbatimbaselineskip' e' uno dei parametri personalizzabili,
% ad esempio:
%
%   \verbatimbaselineskip = .95\verbatimbaselineskip
%
% riduce l'interlinea del testo verbatim del 5% rispetto al valore TeX corrente.
% oppure:
%
%   \verbatimbaselineskip = 10pt plus .1pt minus .1pt
%
% fissa l'interlinea del testo verbatim a 10pt a meno di un aggiustamento positivo o
% negativo pari ad un massimo di .1pt. Si noti che questa liberta' di spaziatura 
% verticale data a TeX sana l'errore di 'underfull \vbox' che altrimento si avrebbe
% ad ogni pagina del testo verbatim.
%
% IMPORTANTE! L'istruzione:
%
%   \catcode`\@ = 0
%
% indica a TeX che all'interno dell'ambiente verbatim il carattere escape e' "@",
% non "\". Per questo motivo occorre utilizzare il comando "@endverbatim" per 
% chiudere il blocco verbatim e non "\endverbatim". 
% In verbatim i comandi sono dunque per TeX tutte e sole le stringhe di caratteri
% che iniziano con il carattere "@" (essi vanno utilizzati pero' con alcune cautele,
% specialmente nei riguardi del trattamento della spaziatura...)
% Un esempio di comando di frequente utilizzo e' '@eject' oppure '@vfill@eject' che
% permettono di passare alla pagina successiva del documento, nel secondo caso dopo
% aver riempito di spazio verticale la parte finale della pagina (nota: non lasciare
% righe vuote dopo questo comando altrimenti esse verranno inserite all'inizio della
% pagina seguente!).
% Ancora, per inserire un file testo esterno si puo' procedere come segue:
%
%    \verbatim[\tentt]
%    @input file.txt 
%    @endverbatim
%
% oppure:
%
%    \def\inputverbatimfile{\input c:/listato.cpp}
%    \verbatim[\tentt]
%    @inputverbatimfile 
%    @endverbatim
%
% Il carattere '@' come detto e' un carattere speciale. Se occorre inserirlo
% all'interno del testo verbatim bisogna utilizzare la coppia di caratteri "@@"
% (vedi definizione '\def\@{@}').
% Consideriamo ora le linee di codice dell'ambiente verbatim:
%
%    \def\verbatim{%
%        \endgraf\bgroup\the\everyverbatimbegin\setupverbatim\verbatimfont}
%    \def\endverbatim{\egroup\removelastbox\endgraf{\the\everyverbatimend}}
%  
% Il comando '\verbatimfont' imposta il tipo di font che TeX utilizzera' nel blocco
% verbatim. Tale definizione e' ovviamente personalizzabile, ad esempio:
%
%    \def\verbatimfont = cmtt9
%
% indica a TeX di utilizzare il font della famiglia '\tt', a 9pt.
% L'istruzione '\removelastbox' indica a TeX di eliminare l'ultimo carattere '^^M'
% del blocco verbatim, che avrebbe l'effetto di introdurre una linea vuota non
% richiesta dopo le linee verbatim.
% In ultimo, i due token: '\everyverbatimbegin' e '\everyverbatimend' permettono di
% rendere automatiche azioni TeX da effettuarsi rispettivamente all'inizio ed alla
% fine del blocco verbatim, ad esempio:
%       
%    \everyverbatimbegin = {
%        % riduzione interlinea del 5% + '\bigskip'
%        \verbatimbaselineskip = .95\verbatimbaselineskip\bigskip}
%        % utilizzo in verbatim del font typewriter (famiglia \tt) a 9pt
%        \font\verbatimfont = cmtt9}
%    \everyverbatimend = {%
%        \medskip\centerline{$\diamond\quad\diamond\quad\diamond$}\bigskip}
%
% permettono rispettivamente l'inserimento di un '\bigskip' prima di ogni blocco
% verbatim e di un segno grafico seguito da '\bigskip' alla fine di ogni blocco.
% Viene anche selezionato il tipo di carattere da utilizzarsi nelle linee di testo
% verbatim (al posto del carattere di default '\tt').
%
% Per un esempio si veda il file 'verbtest.tex'.
%%%%%%%%%%%%%%%%%%%%%%%%%%%%%%%%%%%%%%%%%%%%%%%%%%%%%%%%%%%%%%%%%%%%%%% by MD!99 %%%

\chardef\other = 12
\def\makeactive#1{\catcode`#1 = \active\ignorespaces}
\def\uncatcodespecials{\def\do##1{\catcode`##1 = \other}\dospecials}
\def\removelastbox{\setbox0 = \lastbox}

\def\newline{\endgraf\noindent}                % definizione end-line (modificabile)
{\obeyspaces\gdef\tabtospaces{\ \ \ }}    % un tab equivale a 3 spazi (modificabile)

% azioni da eseguire ad ogni inizio e fine blocco verbatim (default = nessuna)
\newtoks\everyverbatimbegin\everyverbatimbegin = {\relax}           % (modificabile)
\newtoks\everyverbatimend\everyverbatimend = {\relax}               % (modificabile)
% interlinea del testo verbatim (settata al valore corrente)
\newskip\verbatimbaselineskip\verbatimbaselineskip = \baselineskip  % (modificabile)
% font utilizzato nel blocco verbatim (valore di default)
\let\verbatimfont = \tt                                             % (modificabile)

\bgroup          % il gruppo delimita le linee di testo in cui sono attivi ^^M e ^^I
    \makeactive\^^M \makeactive\^^I %
    \gdef\setupverbatim{%
        \makeactive\^^M \let^^M = \newline                       % ridefinizione ^^M
        \makeactive\^^I \let^^I = \tabtospaces            % conversione tab in spazi
        \catcode`\@ = 0 \uncatcodespecials                       % modifica \catcode
        \parskip = 0pt \baselineskip = \verbatimbaselineskip            % interlinea
        \obeyspaces\frenchspacing}                             % spaziatura francese
\egroup
\def\@{@}              % in verbatim il carattere "@" si ottiene con la stringa "@@"

\def\verbatim{\endgraf\bgroup\the\everyverbatimbegin\setupverbatim\verbatimfont}
\def\endverbatim{\egroup\removelastbox\endgraf{\the\everyverbatimend}}
%%%%%%%%%%%%%%%%%%%%%%%%%%%%%%%%%%%%%%%%%%%%%%%%%%%%%%%%%%%%%%%%%%%%%%%%%%%%%%%%%%%%