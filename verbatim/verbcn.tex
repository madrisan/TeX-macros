%%%%%%%%%%%%%%%%%%%%%%%%%%%%%%%%%%%%%%%%%%%%%%%%%%%%%%%%%%%%%%%%%%%%%%%%%%%%%%%%%%%%
% @TeX-file{
%    author            = "Davide MADRISAN",
%    filename          = "verbcn.tex",
%    version           = "3.4b",
%    date              = "16/Mag/1999",
%    copyright         = "This file copyright by Davide Madrisan",
%    codetable         = "ISO/ASCII",
%    checksumtype      = "line count",
%    checksum          = "146",
%    dependencies      = "plain.tex",
%    abstract          = "This package implements a verbatim environment. It is
%                         intended to offer a good interface to programmers that
%                         need to insert verbatim text or files into TeX documents".
% }
%%%%%%%%%%%%%%%%%%%%%%%%%%%%%%%%%%%%%%%%%%%%%%%%%%%%%%%%%%%%%%%%%%%%%%%%%%%%%%%%%%%%

% RESTRICTIONS: the verbatim macros are provided free of charge without warranty of
% any kind. They may be distributed freely provided that the recipients also acquire
% the right to distribute them freely. The notices to this effect must be preserved
% when the files are distributed. If you want to change something, change the 
% filename from 'verbcn.tex' to something else and indicate any changes in the file
% itself, or else remove my name from the file completely.

\def\shouldbefmt{plain}        % controllo precauzionale del formato di compilazione
\newlinechar = `\^^J
\def\lookatformat{                   % this package requires plain TeX to run safety
    \ifx\shouldbefmt\fmtname%
    \else\immediate\write16{^^J%
        > !OOPS... warning! - I'm not running under Plain TeX!^^J}\fi}\lookatformat

% (2) %%% verbatim environment for courier new (True Type) fonts %%%%%%%%%%%%%%%%%%%
% La seguente variante dell'ambiente verbatim implementato nel file 'verbtt.tex'
% permette l'utilizzo del carattere True Type 'courier new' al posto del font
% (orrido) typewriter Computer Modern di TeX.
% Occorre naturalmente avere il file '.tfm' (usato solo in fase di compilazione) ed
% il corrispettivo file '.ttf' (usato nella visualizzazione su video e nella stampa)
% installati nel proprio sistema (in PCTeX32 per creare i file '.tfm' si puo' agire
% tramite menu': 'Settings' --> 'TFM File Names...').
% La sintassi d'uso di questa variante e' identica alla versione implementata nel
% file 'verbtt.tex' (si vedano al proposito i commenti in esso contenuti):
% 
%    \noindent Qualsiasi testo che si richiede venga formattato da \TeX.
%    \verbatim
%    matrix::~matrix() { // distruttore della classe matrix
%       if(--p->n == 0) { // se il contatore di riferimenti diventa 0
%          for(int r = 0; r < rows(); r++)
%             delete p->m[r];
%          delete p->m;
%          delete p;
%       }
%    } 
%    @endverbatim
%    \noindent Ritorno al trattamento di default del testo.
%
% Poiche' la tabella dei caratteri dei font True Type non e' completamente uguale a
% quella dei font Computer Modern, occorre ridefinire alcuni caratteri.
% Tali ridefinizioni devono essere purtroppo globali ma vengono effettuate dopo
% avere modificato il codice carattere ('\catcode') di questi caratteri al valore 13
% (valore che nel formato plain TeX non assumono mai). In questo modo non dovrebbero
% esserci effetti collaterali su blocchi di testo TeX seguenti l'ambiente verbatim
% (io almeno non ne ho riscontrato alcuno nella fase di test delle macro...).
% I valori di default dei '\catcode' vengono ripristinati a fine ambiente verbatim
% tramite la coppia di macro:
%
%    \activecharcodes \recovercatcodes
%
% I parametri modificabili in questa implementazione di verbatim sono:
%
% - \font\verbatimfont: puo' essere scelto un altro carattere (come il courier new
%   non grassetto o un qualsiasi altro carattere a spaziatura fissa) o semplicemente
%   indicata una differente dimensione del font;
% - L'interlinea '\verbatimbaselineskip' (si veda 'verbtt.tex');
% - La macro '\newline' (si veda 'verbtt.tex');
% - La macro '\tabtospaces' puo' essere modificata per tradurre il carattere di
%   tabulazione in un differente numero di spazi;
% - I token '\everyverbatimbegin' e '\everyverbatimend' possono essere modificati
%   (si veda 'verbtt.tex').
%%%%%%%%%%%%%%%%%%%%%%%%%%%%%%%%%%%%%%%%%%%%%%%%%%%%%%%%%%%%%%%%%%%%%%%%%%%%%%%%%%%%

% font utilizzato nel blocco verbatim (valore di default)
% tale identificatore viene utilizzato piu' in basso per modificare la spaziatura)
\font\verbatimfont = courinb at 8pt      % (courier new grassetto 8pt; modificabile)

\chardef\other = 12
\def\makeactive#1{\catcode`#1 = \active\ignorespaces}
\def\activecharcodes{\do\-\do\"\do\<\do\>\do\{\do\}\do\\\do\|}
\def\makespecials{\def\do##1{\catcode`##1 = \active}\activecharcodes}
\def\resetcatcodes{\def\do##1{\catcode`##1 = \the\catcode`##1}\activecharcodes}
\def\uncatcodespecials{\def\do##1{\catcode`##1 = \other}\dospecials}
\def\removelastbox{\setbox0 = \lastbox}

% azioni da eseguire ad ogni inizio e fine blocco verbatim (default = nessuna)
\newtoks\everyverbatimbegin\everyverbatimbegin = {\relax}           % (modificabile)
\newtoks\everyverbatimend\everyverbatimend = {\relax}               % (modificabile)
% interlinea del testo verbatim (settata al valore corrente)
\newskip\verbatimbaselineskip
\verbatimbaselineskip = \baselineskip plus .1pt minus .1pt          % (modificabile)

% ridefinizione spaziatura (si dimensiona in base al font "\cournew" selezionato)
\setbox0 = \hbox{\verbatimfont m} % (una lettera qualsiasi)
\newdimen\spacewd\spacewd = \wd0
\def\spacekern{\kern\spacewd}
{\obeyspaces\global\let =\spacekern}

% nota: \hskip0pt permette la visualizzazione delle linee vuote
\def\newline{\endgraf\noindent\hskip0pt}       % definizione end-line (modificabile)
\def\tabtospaces{\hskip3\spacewd\relax}   % un tab equivale a 3 spazi (modificabile)

\bgroup          % il gruppo delimita le linee di testo in cui sono attivi ^^M e ^^I
    \makeactive\^^M \makeactive\^^I %
    \gdef\setupverbatim{%
        \makeactive\^^M \let^^M = \newline                       % ridefinizione ^^M
        \makeactive\^^I \let^^I = \tabtospaces            % conversione tab in spazi
        \catcode`\@ = 0 \uncatcodespecials\makespecials          % modifica \catcode
        \parskip = 0pt \baselineskip = \verbatimbaselineskip            % interlinea
        \obeyspaces\frenchspacing}                             % spaziatura francese

% correzioni dovute a diversa(!) tavola caratteri ttfonts/cmfonts
% nota 1. nelle linee seguenti vengono ridefiniti in modo GLOBALE alcuni caratteri
% nel caso della categoria di codice (\catcode) 13. Nei testi TeX tali caratteri
% appartengono normalmente a categorie diverse dalla 13, quindi cio' non dovrebbe
% provocare effetti collaterali...
% nota 2. occorre '\relax', altrimenti:
%   -- diventa un trattino lungo;
%   \" provoca 'bad character code error' se seguito da un numero;
%   \  come sopra, (es. '\0').

    \makeactive\- \gdef-{\char'055\relax}
    \makeactive\" \gdef"{\char'175\relax}
    \makeactive\| \gdef|{\char'227\relax}
    \makeactive\< \gdef<{\char'241\relax}
    \makeactive\> \gdef>{\char'277\relax}
    \catcode`\< = 1                         % '<' e' ora il simbolo di inizio gruppo
    \catcode`\> = 2                         % '>' e' ora il simbolo di fine gruppo
    \makeactive\{ \gdef{<\char'226\relax>   % nuova def. globale di '{' (codice 13)
    \makeactive\} \gdef}<\char'224\relax>   % nuova def. globale di '}' (codice 13)
    \catcode`\@ = 0                         % '@' e' ora il car. escape di TeX
    \makeactive\\ @gdef\<@char'223@relax>   % nuova def. globale di '\' (codice 13)
@egroup              % chiusura blocco: ripristino dei valori dei codici di -"<>{}@\

\def\@{@}              % in verbatim il carattere "@" si ottiene con la stringa "@@"

\def\verbatim{\endgraf\bgroup\the\everyverbatimbegin\setupverbatim\verbatimfont}
\def\endverbatim{\egroup\removelastbox\endgraf\resetcatcodes{\the\everyverbatimend}}
%%%%%%%%%%%%%%%%%%%%%%%%%%%%%%%%%%%%%%%%%%%%%%%%%%%%%%%%%%%%%%%%%%%%%%%%%%%%%%%%%%%%